% Chapter 6

\chapter{Related Work} % Main chapter title
\label{Related Work} % For referencing the chapter elsewhere, use \ref{Chapter1} 
The related work section presented in this thesis, focuses on the state of the art in Cloud Computing in regard to it's monitoring, scheduling and SLA management capabilities. Therefore this chapter was divided into five categories, namely: (1) Infrastructure Measurements and Cloud Monitoring, (2) SLA Description Languages, (3) Performance Prediction, (4) Scheduling Mechanisms and (5) SLA Enforcement.

\section{Infrastructure Measurements and Cloud Monitoring}
Cloud Computing today delivers nearly unlimited scalable on demand computing resources within a few clicks. But the monitoring and government of such resources may come with some requirements. Monitoring is one of the fundamental parts of every cloud platform, since monitoring is needed to scale applications or instances correctly based on their resource utilization. It is also need to detect and discover defects and limitations, such as bottlenecks in the infrastructure environment. Additionally monitoring can give viable insights to both Cloud users and providers by revealing usage patterns and trends. Without any form of monitoring Cloud providers would not even be able to invoice their customer correctly, since they would not be able to tell how much ressouces and for how log the customer has used them. Cloud Monitoring has many ties into different Cloud Management fields such as capacity and resource planning and management, SLA management, incident management and troubleshooting, performance management and billing. An early example towards distributed resources monitoring is NetLogger \cite{DBLP:conf/mascots/2000}, a distributed monitoring system, which could monitor and collect network information. Published in 2000, application could use NetLoggers API to gather load information of the network and react accordingly. With GridEye \cite{Fu:2006:GSG:1170138.1170724} a service-oriented monitoring system, with an flexible, scaleable architecture and forecasting algorithm for performance prediction on the basis of traditional MA(k) and ExS(alpha) models was purposed. In 2009 Sandpiper \cite{Wood:2009:SBG:1663647.1663710} was proposed, a system, which automatically could monitor and detect hotspots and based on that remap or reconfigure VMs if necessary. The described system marks a first step towards autonomous SLA management, since the internal remaping algorithm considered SLA violations and to avoid them. 
In recent years various academic and commercial Cloud monitoring solutions have been proposed. In 2014 Jonathan Stuart Ward and Adam Barker \cite{ward2014observing} published a taxonomy of Cloud Monitoring stating scalability, cloud awareness, fault tolerance, aromaticity, comprehensiveness, timeliness and multiply granularity as core requirements towards effective Cloud monitoring. Fatema et al. \cite{fatema2014survey} and Aceto et al. \cite{aceto2013cloud} analized the most common open source and commercial monitoring solutions, such as Nagios, Collectd, Ganglia, IBM Tivoli, Amazon Cloud Watch, Azure Watch, PCMONS, mOSAIC, CASViD and many more, according on how they relate with these requirements. Recently Manasha Saqib \cite{saqib2017cloud} apportioned these requirements along the seven layers of Cloud Computing accoring to \cite{CSA3.0} \cite{spring2011monitoring1} \cite{spring2011monitoring2}: Facility, Network, Hardware, Operating System (OS), Middleware, Application and User. These studies provided a comprehensive overview of the available monitoring tools and their ability to support Cloud operational management, but also stated their shortcomings in the different areas. Cloud Monitoring in this thesis is considered as an enablement technology. The proposed architecture and algorithms are able to work with different monitoring solutions. The main focus thereby lies within the acquisition and enhancement ot such methodologies with additional information.
An additional topic especially in the area of PaaS is the application monitoring and application performance monitoring in the cloud.The definition of application performance management by Menasce \cite{menasce2002load}: APM is a collection of management processes used by organizations to ensure that the QoS of their e-business applications meets the business goals. So APM directly aims at the application lifecycle and management processes such as monitoring, resource management, performance-management, reporting and so on, of software systems. According to Gartner \cite{GartnerAPM} there are severall established big names in APM such as IBM, Oracle, BMC and so on, that are getting challenges by innovative start-ups such as AppDynamics or Dynatrace, and Correlsense. APM can be used as performance analysis and monitoring agents in SaaS and PaaS, and as such will be an topic for future work.


\section{SLA Description Languages}
NIST \cite{Liu2011} has also pointed out the necessity of SLAs, SLA management, definition of contracts, orientation of monitoring on Service Level Objects (SLOs) and how to enforce them. However a clear definition of a reference of a specific format is missing. This is also the case with the Internet Engineering Task Force (IETF) \cite{Khasnabish2010}. Besides these approaches of the companies and organisations there are further efforts to develop and to realize cloud management architectures and systems. A basic discussion can be found in the book from Wieder et.al. \cite{Wieder2011} mainly concerned about SLA definitions and negotiations.

In 2003 IBM developed the Web Service Agreement Language (WSAL) \cite{Ludwig03WSLA} which is still available in the Version 1.0 that dates back to the same year. It has not been further developed since. WSAL focused on performance and availability metrics. It seriously lacks expression therefore it is not powerful enough. The required flexibility is also missing which is needed for dynamical changes at run time. It is closely connected to their XML schema and not very useful to determine conclusion.
WSAL has been mainly developed for Web services, its usage in other fields is questionable. It shows significant shortcomings regarding content as it focusses mainly on technical properties. The structural requirements, however, are met as discussed in Spillner et.al. \cite{Spillner2009}.

The WS-Agreement (WS-A) was developed by the Open Grid Forum in the year 2007 (Version 1.0) The last update which was based on the work of the European SLA@SOI project was done in 2011. Its advantages are the expandability and the adaptability which is, on the other hand, also one of its greatest disadvantages because it has not been specified in details by Kearney \cite{Kearney2011b}. It is based on technical transformation, the structural transformations, however, have not been taken into account.

The Foundation of Self Governing ICT Infrastructures (FOSll)-Project \cite{fosii} is another research project which aims at the usage of autonomic principals for information and communication systems. Self determining infrastructures should be realized and made available through cloud based services. Within the LoM2HiS autonomic SLA management is realized by translation of system parameters to abstract KPIs and SLOs  \cite{Brandic:2009:VFE:1616056.1616063}. The SLA specification is based on WSAL. The Texo project \cite{texo2011} attempts within the THESEUS research program to realize a conception of service descriptions, contract management, end to end marketplaces and monitoring from a business perspective. In addition the development and use of WS-A based SLAs are needed. Looking at the cloud interfaces that describe the management of resources within the cloud, it becomes obvious that they are exclusively designed with the purpose of system monitoring. They do not provide a direct monitoring of SLAs. Therefore the placement of machine readable SLAs is extremely difficult. This is also the case with existing monitoring tools.

SLA handling in clouds, resulting from the EU project OPTIMIS, is discussing negotiating and creating Service Level Agreements between infrastructure providers and service providers \cite{Lawrence:2010:USL:2050107.2050112}.
Although it is enhanced within the SLA@SOI project \cite{slasoi2011} its development is unclear because the SLA@SOI project develop its own format SLA@SOI SLA(T) \cite{slasoiwiki} and supported by the European IT industry. A comprehensive project result has been published on their web page. No independent analysis of the advantages or disadvantages of the SLA(T) format is available at the moment. It provides all structural requirements of a SLA and it has the greatest intersection with regard to content. SLA(T) is the basis of the proposed approach in this paper. Further it should be accentuated that a meta model SLA* \cite{slasoisrc} is defined which simplifies the extension and adaptability for SLA(T).

\section{Performance Prediction}
Neural Networks are widely used in forecasting problems. One of the earliest successful application of ANNs in forecasting is reported by Lapedes and Farber \cite{LapedesFarber87}. They used a feed-forward neural network with deterministic chaotic time series generated by the Glass-Mackey equation, to predict such dynamic non-linear systems.
Artificial Neural Networks are proven universal approximators \cite{Hornik1989359}\cite{Hornik1991251} and are able to forecast both linear \cite{Zhang20011183} and non-linear time series \cite{Zhang199835}. Adya and Collopy investigated in the effectiveness of Neural Networks (NN) for forecasting and prediction \cite{neural1}. They came to the conclusion that NN are well suited for the use of prediction, but need to be validated against a simple and well-accepted alternative method to show the direct value of this approach. Since forecasting problems are common to many different disciplines and diverse fields of research, it is very hard to be aware of all the work done in this area.  Some examples are forecasting applications such as: temperature and weather \cite{Langella2010328}\cite{Buizza}\cite{Roebber}, tourism \cite{Pattie1996151}, electricity load \cite{Park76685}\cite{Hippert910780}, financial and economics \cite{Bodyanskiy20061357}\cite{McAdam2005848}\cite{Kaastra1996215}\cite{Guresen201110389} and medical \cite{Vukicevic2014}\cite{Arizmendi20145296} to name a few. Zhang, Patuwo, and Hu \cite{Zhang199835} show multiple other fields where prediction by ANN was successfully implemented.

%While NNs powerful approximation capabilities and selfadaptive data driven modelling approach allow them great flexibility in modelling time series data, it also complicates substantially model specification and the estimation of their parameters. Direct optimisation through conventional minimisation of error is not possible under the multilayer architecture of NNs and the back- propagation learning algorithm has been proposed to solve this problem (Rumelhart, Hinton, & Williams, 1986), later discussed in the context of time series by Werbos (1990). Several complex training (optimisation) algorithms have appeared in the literature, which may nevertheless be stuck in local optima (Hagan, Demuth, & Beale, 1996; Haykin, 2009). To alleviate this problem, training of the networks may be initialised several times and the best network model selected according to some fitting criteria. However, this may still lead to suboptimal selection of parameters depending on the fitting criterion, resulting in loss of predictive power in the out-of-sample set (Hansen & Salamon, 1990). Another chal- lenge in the parameter estimation of NNs is due to the uncertainty associated with the training sample. Breiman (1996a) in his work on instability and stabilization in model selection showed that sub- set selection methods in regression, including artificial neural net- works, are unstable methods. Given a data set and a collection of models, a method is defined as unstable if a small change in the data results in large changes in the set of models.

Similar research with different focus has been conducted in the past for the use of machine learning in cloud environments. Prevost et al. used a Neural Network (NN), as well as a Linear Predictor \cite{prevost2011prediction} to anticipate future workloads by learning from historical URL requests. Although both models were able to give efficient predictions, the Linear Predictor was able to predict more accurately. Li and Wang proposed their modified Neural Network algorithm nn-dwrr in \cite{li2014sla}. The application of this algorithm led to a lowered average response time compared to application of traditional capacity based algorithms for scheduling incoming requests to VMs. In similar research Hu et al. \cite{hu2013kswsvr} have shown that their modification of a standard Support Vector Regression (SVR) algorithm can lead to an accurate forecasting of CPU Load what can be used to achieve a better resources utilization. Another algorithm, which is renowned for providing good results in similar scenarios, is Linear Regression. Although the results are often weaker compared to Neural Networks or Support Vector Machines (SVM) in cases of workload prediction \cite{bankole2013predicting} \cite{imam2011neural}, the fast training and deployment time of models built with Linear Regression should not be underestimated. Those examples show that there are a variety of optimization challenges in cloud environments which can be tackled by applying machine learning algorithms. What separates the current work from previous research is a detailed examination of specific characteristics of three different machine learning algorithms and presenting the results in a visual way. The choice to evaluate Neural Networks, Support Vector Machines and Linear Regression was made because those algorithms earned promising results in previously conducted research.


\section{Scheduling Mechanisms}

Haizea \cite{haizea2} is a resource manager or resource scheduler. Haizea  can manage a set of computers (typically a cluster), allowing users to request exclusive use of those resources described in a variety of terms, such as "I need 10 nodes, each with 1 GB of memory, right now" or "I need 4 nodes, each with 2 CPUs and 2GB of memory, from 2pm to 4pm tomorrow". Haizea provides reservation and deadline sensitive type of leases along with the traditional immediate and best effort policies \cite{haizea1} \cite{haizea3} \cite{haizea4}. Haizea uses simple allocation policies for deadline sensitive leases. It tries to find out a single slot of required time between startTime and endTime of the given lease which can allocate the requested resources. If it is unable to find such a time slot, it rejects the lease. 

\section{SLA Enforcement}
SLA@SOI
SLAs-LoM2HiS framework
Slaws
\section{Autonomic Computing}
FoSIIResearchProject

%----------------------------------------------------------------------------------------
% links and stufff
%
%
%http://www.sla-ready.eu/ SLA Ready EU Projekt
%http://www.cloudwatchhub.eu/
%
%http://www.specs-project.eu/
%
%ISO/IEC CD 19086-1  Information technology -- Cloud computing -- Service level agreement (SLA) framework and Technology -- Part 1: Overview and concepts
%
%ISO/IEC NP 19086-2 Information technology -- Cloud computing -- Service level agreement (SLA) framework and Technology -- Part 2: Metrics
%
%ISO/IEC 27004:2009 Information technology -- Security techniques -- Information security management -- Measurement
%
%ISO/IEC FDIS 27017 Information technology -- Security techniques -- Code of practice for information security controls based on ISO/IEC 27002 for cloud services
%
%NIST Cloud Computing Service Metrics Description http://www.nist.gov/itl/cloud/upload/RATAX-CloudServiceMetricsDescription-DRAFT-20141111.pdf
%
%Standards terms and performance criteria in service level agreements for cloud computing services - http://www.sla-ready.eu/sites/default/files/Finalreport.pdf

