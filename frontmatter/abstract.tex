%
% File: abstract.tex
% Author: V?ctor Bre?a-Medina
% Description: Contains the text for thesis abstract
%
% UoB guidelines:
%
% Each copy must include an abstract or summary of the dissertation in not
% more than 300 words, on one side of A4, which should be single-spaced in a
% font size in the range 10 to 12. If the dissertation is in a language other
% than English, an abstract in that language and an abstract in English must
% be included.
\chapter*{COPYRIGHT STATEMENT}
\begin{SingleSpace}
This copy of the thesis has been supplied on condition that anyone who consults it is understood to recognise that its copyright rests with its author and that no quotation from the thesis and no information derived from it may be published without the author's prior consent.
\end{SingleSpace}
\clearpage


\chapter*{Abstract}
\begin{DoubleSpace}
\textbf{Autonomic Management of Service Level Agreements in Cloud Computing}

Cloud computing has been one of the major trending topics of recent years in the Information Technology (IT) industry. In contrary to the previous IT service delivery models, where all the services and resources where hosted locally, the idea behind cloud computing is to deliver computing resources or services on-demand over a network on an easy pay-per-use business model. This paradigm change included almost all areas of the modern IT landscape, be it the storage and sharing of data for example with services like Dropbox, Microsoft OneDrive or Google Drive or creating complete computational environments with Amazon Web Services or the Microsoft Azure Cloud Platform. Due to the lower upfront costs, rapid provisioning, elasticity and scalability, the adoption of cloud services is steadily increasing. Thereby cloud computing nowadays for many companies has become a practical alternative to locally hosted resources and IT services. However, in the currently offered state of cloud computing, there are significant shortcomings in regard to guarantees and the Quality of Service (QoS). In order to make cloud services effectively usable and reliable for enterprises, Service Level Agreements (SLA) are needed, which state the precise level of performance, as well as the manner and the scope of the service provided. This practice, which is widespread in the area of IT services, but is currently of limited use for cloud computing, due to the fact that existing cloud environments offer only rudimentary support and handling of SLAs, if any. Moreover, the classic SLA management approach is a rather static method, whereas due to the dynamic character of the cloud, the QoS attributes respectively service levels must be monitored and managed continuously. In addition, performance indicators and measurement methods for service level objectives in cloud computing have been studied inadequately. The proposed research focuses on cloud specific SLA management as well as the associated functional and non- functional QoS parameters and measurement methods. To address the shortcomings of the current state of the art cloud computing SLA landscape and enable cloud users to dynamically create, adjust and monitor SLAs for their cloud services, an architecture for providing autonomous management of cloud services together with an easy to use SLA control interface is introduced, to monitor the agreed service performance, facilitate SLA compliance and enable dynamic customer- specific adaptation of SLAs for cloud services. Together with the specially tailored machine readable Adaptable Service Level Objective Agreement language (A- SLO-A ), and highly the specialized scaling mechanisms, as well as a SLA scheduling module, the proposed research builds a holistic approach of enabling QoS and SLAs in Cloud Computing.
\end{DoubleSpace}
\clearpage


